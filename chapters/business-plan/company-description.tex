\chapter{Company description}
\section{Organizational chart}\label{org-chart}
Through previous endevors, research and talks with various people which work in different management positions, we've come to this organizational chart. The diagram \ref{fig:organisational-chart} depicts all required departments and their teams as well as management positions. Division are marked with the same color. The color's saturation denotes the hierarchical position within said department.

\begin{figure}[!ht]
  \centering
  \includegraphics[width=\linewidth]{./images/organisational-chart.png}
  \caption[Organizational chart made with lucidchart.com]{Diagram of the organization}
  \label{fig:organisational-chart}
\end{figure}

This diagram might be more detailed or complex when compared to ones from different startups, but this is by design. It is important that our service works reliably and follows industry standards. To achieve that, we structurally trade a bit of agility for reliability and structure.

Not displayed are ways or means of communication between teams and/or departments. Also, in the future there might be an additional team, which's job is to ignore department barriers and work on different tasks or enable better communication or a better work flow, depending on the workload of a department.

The following chapters \ref{org-management} through \ref{org-logistics} will now go into further detail on what each department entails and of what teams it is made of.
\subsection{Management}\label{org-management}
The management division is marked with the color green. This diagram also includes the shareholders and the executive board. This was done in order to paint a complete picture about the company and who has a say in key decisions.
\newline
Shareholders want to maximize their profit. Keeping them safe and happy is detrimental for the long term success of the company.
\newline
The executive board's and the \acs{ceo}'s job is to lead the company to success and thus growth and profitability. They, especially the \acs{ceo}, have the big picture in mind and guide the departments to said goals. He or she has a say in both the "Technical Division" and the "Commercial \& Controlling Division".
\newline
\newline
If uncertainties arise, consulting is also an option to get an external perspective.
\subsection{\acl{rnd}}\label{org-rnd}
This department is marked with the color indigo.
\newline
By choice, we've structured this technical department into two teams. Both are concerned with "\acl{rnd}", but one focuses on software while the other focuses on hardware. It's detrimental to us, that both hard and software is done in-house. Having the software team in-house instead of offshore increases the quality, general understanding and agility of our software. Also having hardware in-house enables these teams to work closely together, which further increases the products (and thus also the services) quality and may also lead to a better work environment, as employees are enabled to innovate and take ownership of their own work. \cite{wang_2022_employee}
\subsection{\acl{pno}}
The "\acl{pno}" department, marked with violet, is made of a less specialized workforce, which focuses on supporting other departments, customers or general processes.
\subsection{Accounting}
The accounting department is featured with a fuchsia color. It encapsulates a general sales and a marketing team. These are separated and under the same department by design. Their work may be different, but they need to cooperate with each other frequently, which would be harder if they were in different departments.
Additionally, the "Finance and accounting" team is also under this department. It handles all monetary things.
\subsection{Administration}
The red administration department features a classical "Human resources" team, as well as internal paramedics and the "Market research" team. The last one is concerned with the development of the overall market we operate in and how we maneuver in it. It's findings are detrimental, reported to the CEO, which in turn influences the roadmap of the \ac{rnd} department.
\subsection{Logistics}\label{org-logistics}
Lastly the logistics department marked with orange. The decision to produce both soft- and hardware in-house requires a bigger logistical effort than if it was done offshore. One team will overlook the supply chain, so that if a global supply bottleneck occurs, we're going to be the least impacted as possible. They're also tasked with prioritizing suppliers which care about sustainability. Two other teams are concerned with warehousing of material while another looks after transportation of goods.
\section{Vision}
\begin{quote}
  \say{\emph{Build a world where accessible, sustainable and fast medical care is the standard.}}
\end{quote}

We want to bring a change to medical care and research. The technological advancements can and must be used in this field to give patients the best care they can receive. No dying patient should have to wait unnecessarily for an organ transport. No treatment should be delayed because of traffic jam. There is no reason for a two ton vehicle for transportation, if one or several drone/-s can get the same job done.
\newline
\newline
Let's be the change that we need.

\section{Mission}
\begin{quote}
  \say{\emph{Reliably support professionals in providing critical health care through innovative means of transportation.}}
\end{quote}

How we want to achieve our vision is through supporting key figures in health care industry.
\begin{enumerate}
  \item Improve the transportational processes of companies doing research, which would lead to faster time-to-market of new and improved medicines and treatment plans.
  \item Fast, uncomplicated and reliable transport of timely good like organ or blood donations.
  \item Fast, uncomplicated and reliable transport of lab probes or results
  \item Easily and safely accessible medicines for elderly, handicapped, etc.
  \item Lessen the load on streets and pollution on the environment
\end{enumerate}

\section{Contributions to sustainability}
We have three means on how we can add our contribution to sustainability. Each of the following chapters will tackle one of our strategies.
\subsection{Full control over the drones' life cycle}
As mentioned in chapters \ref{org-chart}, \ref{org-rnd} and \ref{org-logistics}, we have full control over not just the service layer of the drones but also the production and logistics that go with it.
\newline
Having this in house allows us to improve processes and make them more carbon efficient. Also, having a dedicated "Supply Chain Management" team allows us to compare suppliers and choose the ones best suited for our mission. Furthermore, the transportational efforts are exponentially less, if everything is done under one roof.
\newline
We also have the advantage, that we are allowed make investments into sustainability, as costs can be spread out over many deliveries for many businesses.
\newline
Lastly, one of our core values is a good work environment where people are encouraged to grow and express themselves. This leads to happier employees and thus has a positive impact on the local community and society as a whole.
\subsection{Precision of transportation}
Conventional transportation of goods requires big busses loaded with many packages to be profitable. Our service offers the solution of direct and precise deliveries. Thus, we are not dependent on things like traffic, other packages, which lessens the overhead of CO2 production. Routing a delivery vehicle to many drop off points will be less efficient than direct deliveries. Also, a vehicle would waste gas during traffic jams or slow moving traffic.
\subsection{Electricity}
Compared to conventional transportation, our drones use electricity to transport goods. This makes them, after production, more environmentally sustainable. Usually, this claim enables companies to use this claim for publicity while offloading the carbon footprint to an energy company. This way there might still be coal burned to fuel our drones with electricity.
\newline
We want to go the extra mile and buy all or at least a majority percentage of the used power from renewable energy sources.
